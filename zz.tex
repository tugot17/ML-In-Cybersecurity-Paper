Nowadays we are surrounded by technology. We quickly adapt technological innovations and start to overtrust it. We forget it may fail either because of unpredictable use case or because of malicious attack. Sharif et al. shown that even solutions believed to be  state-of-the-art, may be successfully attacked. Their research proved that cheap solutions that nearly anyone can prepare at home may be threat to sophisticated public security systems. They shown that eyeglass frames printed at simple home printer may obtain really good results at committing biometric frauds. Because of these threats, the machine-learning community should continue to work to ensure security of biometric systems. Hopefully further works of researchers will focus on creating systems robust to attacks for safer future for all of us. 

For me personally, it was the first meeting with the combination of machine learning and cybersecurity, which was a very interesting experience. After reading this article, I will definitely be less trusting about biometric security and will be aware that they are not unbreakable.